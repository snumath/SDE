\documentclass[12pt]{report}
\usepackage[left=2cm,right=2.6cm,top=2.6cm,bottom=3cm,a4paper]{geometry}
\renewcommand{\baselinestretch}{1.2}

\usepackage[utf8]{inputenc}
\usepackage{fancyhdr}
\usepackage{lastpage}
\usepackage{extramarks}
\usepackage[inline]{enumitem}
\usepackage{linegoal}
\usepackage{amsmath,amssymb,latexsym,amsfonts, amsthm}
%\usepackage{verbatim}
%\usepackage{xcolor}
%\usepackage{listings}
\usepackage{comment}
\usepackage{mathtools}
\usepackage{kotex}

%\usepackage[math]{iwona}

\usepackage[tracking]{microtype}
\usepackage[sc,osf]{mathpazo}

\usepackage[all]{xy}


\usepackage{tikz}
\usepackage{tikz-cd}
\usetikzlibrary{arrows}
\usetikzlibrary{matrix}


%%%%%%%%%%%%%%%%%%%%%%%%%%%%%%%%%%%%%%%%%
% New command
\newcommand{\R}{\mathbb{R}}
\newcommand{\RP}{\mathbb{RP}}
\newcommand{\Z}{\mathbb{Z}}
\renewcommand{\r}{\rightarrow}
\newcommand{\g}{\gamma}
\newcommand{\E}{\mathcal{E}}
%\renewcommand{\S}{\mathcal{S}}
%\newcommand{\T}{\mathcal{T}}
%\newcommand{\M}{\mathfrak{M}}

%\newcommand{\inft}{\int_{-\infty}`^\infty}
\newcommand{\rk}{\text{rank }}

\renewcommand{\subset}{\subseteq}
\renewcommand{\supset}{\supseteq}
%\newcommand{\ri}{\Rightarrow}
\newcommand{\bigslant}[2]{{\raisebox{.1em}{$#1$}\left/\raisebox{-.1em}{$#2$}\right.}}

\newcommand{\RNum}[1]{\uppercase\expandafter{\romannumeral #1\relax}}


%%%%%%%%%%%%%%%%%%%%%%%%%%%%%%%%%%%%%%%%%
% Theorem 지정해주는 곳

\newtheoremstyle{break}
{\topsep}{\topsep}%
{\itshape}{}%
{\bfseries}{}%
{\newline}{}%
\theoremstyle{break}
\newtheorem{thm}{Theorem}[chapter] % reset theorem numbering for each chapter

\newtheoremstyle{newdef}% name
{}%         Space above, empty = `usual value'
{}%         Space below
{}%         Body font
{}%         Indent amount (empty = no indent, \parindent = para indent)
{\bfseries}%Thm head font
{}%        Punctuation after thm head
{\newline}% Space after thm head: \newline = linebreak
{}%         Thm head spec
\theoremstyle{newdef}
\newtheorem{defn}[thm]{Definition} % definition numbers are dependent on theorem numbers
\newtheorem{pro}[thm]{Proposition}
\newtheorem{cor}[thm]{Corollary}

\theoremstyle{remark}
\newtheorem*{exmp}{Example} % no numbering example
\newtheorem*{lem}{Lemma}
\newtheorem*{rem}{Remark} % no numbering remark



%%%%%%%%%%%%%%%%%%%%%%%%%%%%%%%%%%%%%%%%%

\begin{document}
\begin{titlepage}
\centering
{\scshape\LARGE Seoul National University \par}
\vspace{1cm}
{\scshape\Large Lecture Note\par}
\vspace{5.5cm}
{\huge\bfseries Introduction to Stochastic \\Differential Equations\par}
\vspace{1.5cm}
\large Lecture by Seo Insuk \\
Notes taken by Lee Youngjae


\vfill
\vspace{1cm}\par
{\large \today\par}
\end{titlepage}

\setlength{\parindent}{0cm}


\setcounter{chapter}{-1}

\chapter{Introduction}
E-mail: \textit{insuk.seo@snu.ac.kr, 27-212}

Grading
\begin{itemize}
\item Mid-terms 1 (15\%, 10/10 or 17)
\item Mid-terms 2 (15\%, 11/7)
\item Fianl-term (40\%)
\item Assignment (20\%, 8-10 times)
\item Attendance (10\%, absent: -2\%, late: -1\%)
\end{itemize}


Let $X$ be a standard normal random variable in $\R$.
i.e., $\mathbb{P}[X \in [a,b]] = \int_a^b \frac{1}{\sqrt{2\pi}} e^{-x^2/2}dx$.
(Central Limit Theorem) If $x_1, x_2, \cdots, x_n \in X, E(x_i) = m, Var(x_i) = \sigma^2$, then
$$
\frac{\frac{x_1-m}{\sigma}+\frac{x_2-m}{\sigma}+\cdots+\frac{x_n-m}{\sigma}}{\sqrt n} \rightarrow X
$$

In this class, we study dynamic version of this theorem.
If $(W_t)_{t\geq 0}$ be a fluctuation, then $(W_t)_{t\geq 0}$ be a random variable in $C[0,T]$

\begin{exmp}
$\frac{dX_t}{dt} = rX_t; dX_1 = rX_tdt$. Then, $X_t = X_0 e^{rt}$ (unrisky assets, bank)\\
$dX_t = rX_tdt + \sigma X_tdW_t$, $\sigma:$ volatility (risky assets, stock)
\end{exmp}

We will study:
\begin{enumerate}
\item Probability Space
\item Random Variable
\item Expectation
\end{enumerate}

Textbooks:
\begin{enumerate}
\item Stochastic Calculus for Finance \RNum{2} (Shreve), covering chapter 1-3 or 4
\item Introduction to Stochastic Integration (Hui-Hsiung Kuo)
\end{enumerate}



\chapter{Probability Space}
There are three elements consisting probability space:
\begin{itemize}
\item $S$: Sample space
\item $\mathcal{E}$: Family of events $E \subset S$ ($\sigma$-algebra in measure theory)
\item $\mathbb{P}$: probability $\Rightarrow \mathbb{P}(E)$ is defined for all $E \subset \mathcal{E}$ ($\mu$ with $\mu(S)=1$)
\end{itemize}

\begin{exmp}
\begin{minipage}[t]{\linegoal}
\begin{enumerate}
\item Toss a coin twice (H for Head, T for Tail)\\
Then, $S = \{$HH, HT, TT, TT$\}$
\item Uniform random variable in $[0,1]^3$\\
Then, $S = [0,1]^3$.
If $E = [0,\frac{1}{2}]^3$, then $\mathbb{P}(E) = Vol(E) = \frac{1}{8}$\\
\end{enumerate}
\end{minipage}

How to define $\mathcal{E}$?\\
In example 2, let $\mathcal{E} = $ family of all subsets of $[0,1]^3$ naively.
But Banach-Tarski Paradox says there are disjoint sets $E,F$ with $\mathbb{P}(E\cup F) \neq \mathbb{P}(E) + \mathbb{P}(F)$ in this $\mathcal{E}$.
Therefore we cannot naively set $\mathcal{E}$ (Use measure theory)\\

In example 1, suppose that we cannot see the second flip.
If $\{HH\} \not\in \mathcal{E}$ and $\{HT, HH\}\in\mathcal{E}$, then $\mathcal{E} = \{\phi, \{HH,HT\}, \{TH,TT\}, \{HH,HT,TH,TT\}\}$
\end{exmp}


\begin{defn}[Measure]
Let $\Omega$ be non-empty set and $\mathcal{F}$ be family of subsets of $\Omega$ with
\begin{enumerate}
\item $\phi \in \mathcal{F}$
\item $A \in \mathcal{F} \Rightarrow A^C \in \mathcal{F}$
\item $A_1, A_2, \cdots \in \mathcal{F} \Rightarrow \bigcup_{i=1}^\infty A_i \in \mathcal{F}$.
\end{enumerate}
We say $\mathcal{F}$ as $\sigma$-albegra or $\sigma$-field, $A\subset \mathcal{F}$ as measurable, and $\Omega$ as measurable space.
\end{defn}


\textit{Exercises.}
\leavevmode
\begin{enumerate}[label = \arabic*)]
\item $\Omega \in F$
\item $A_1, A_2, \cdots \in \mathcal{F}$, then $A_1\cap A_2 \cdots \in \mathcal{F}$
\item $A_1, A_2, \cdots \in A_n \in \mathcal{F}$, then $A_1 \cup \cdots \cup A_n, A_1 \cap \cdots \cap A_n \in \mathcal{F}$.
\item $A, B \in \mathcal{F}$, then $A-B \in \mathcal{F}$
\end{enumerate}


\begin{defn}[Topological Space]
(See Rudin: \textit{Real and Complex Analysis, Chapter 1.})
Let $\Theta$ be non-empty set and $\tau$ be family of subsets of $\Theta$ with
\begin{enumerate}
\item $\phi, \Theta \in \tau$
\item $V_1, \cdots V_n \in \tau \Rightarrow V_1 \cap \cdots \cap V_n \in \tau$
\item $V_\alpha \in \tau \enspace \forall \alpha \in I \Rightarrow \bigcup_{\alpha\in I}V_\alpha \in \tau$.
\end{enumerate}
We say $V\in\tau$ be \textbf{open set}, and $(\Theta,\tau)$ be \textbf{topological space}.
\end{defn}


\begin{defn}[Measurable Function]
$f : (\Omega, \mathcal{F}) \rightarrow (\Theta, \tau)$ is \textbf{measurable} if
$f^{-1}(V) \in \mathcal{F} \enspace \forall V \in \tau$
\end{defn}


\begin{defn}[Positive Measure]
Let $\Omega$ be non-empty set and $\mathcal{F}$ be $\sigma$-algebra.
Then $\mu: \mathcal{F} \rightarrow [0,\infty]$ is called \textbf{measurable} if
\begin{enumerate}
\item $A_1, A_2, \cdots$: disjoint members of $\mathcal{F} \Rightarrow \mu(A_1\cup A_2\cup \cdots) = \sum_{i=1}^\infty \mu(A_i)$
\item $\mu(A) < \infty$ for some $A \in \mathcal{F}$,
\end{enumerate}
and $(\Omega, \mathcal{F}, \mu)$ is called \textbf{measrue space}.
\end{defn}


\begin{defn}
\leavevmode
\vspace{-6mm}
\begin{enumerate}
\item $(\Omega, \mathcal{F}, \mathbb{P})$ is called as \textbf{probability space} if $\mathbb{P}(\Omega) = 1$.
\item $X$ is called as \textbf{random varaible} if it is a function from $(\Omega, \mathcal{F}, \mathbb{P})$ to $\mathbb{R}$
\end{enumerate}
\end{defn}

\underline{Next Class}
\begin{itemize}
\item Borel sets on $\mathbb{R}$ or $\mathbb{R}^d$
\item Lebesgue Measure
\item Lebesgue Integral (Define Expectation of random variable)
\end{itemize}



Some overleaf test

\end{document}
