\documentclass[12pt]{report}
\usepackage[left=2cm,right=2.6cm,top=2.6cm,bottom=3cm,a4paper]{geometry}
\renewcommand{\baselinestretch}{1.2}

\usepackage[utf8]{inputenc}
\usepackage{fancyhdr}
\usepackage{lastpage}
\usepackage{extramarks}
\usepackage[inline]{enumitem}
\usepackage{linegoal}
\usepackage{amsmath,amssymb,latexsym,amsfonts, amsthm}
%\usepackage{verbatim}
%\usepackage{xcolor}
%\usepackage{listings}
\usepackage{comment}
\usepackage{mathtools}
\usepackage{kotex}

%\usepackage[math]{iwona}

\usepackage[tracking]{microtype}
\usepackage[sc,osf]{mathpazo}

\usepackage[all]{xy}


\usepackage{tikz}
\usepackage{tikz-cd}
\usetikzlibrary{arrows}
\usetikzlibrary{matrix}

% tableofcontents with clickable
\usepackage{color}
\usepackage{hyperref}
\hypersetup{
    colorlinks=true, % make the links colored
    linkcolor=blue, % color TOC links in blue
    urlcolor=red, % color URLs in red
    linktoc=all % 'all' will create links for everything in the TOC
}



%%%%%%%%%%%%%%%%%%%%%%%%%%%%%%%%%%%%%%%%%
% New command
\newcommand{\R}{\mathbb{R}}
\newcommand{\RP}{\mathbb{RP}}
\newcommand{\Z}{\mathbb{Z}}
\renewcommand{\r}{\rightarrow}
\newcommand{\g}{\gamma}
\newcommand{\E}{\mathcal{E}}
%\renewcommand{\S}{\mathcal{S}}
%\newcommand{\T}{\mathcal{T}}
%\newcommand{\M}{\mathfrak{M}}

%\newcommand{\inft}{\int_{-\infty}`^\infty}
\newcommand{\rk}{\text{rank }}

\renewcommand{\subset}{\subseteq}
\renewcommand{\supset}{\supseteq}
%\newcommand{\ri}{\Rightarrow}
\newcommand{\bigslant}[2]{{\raisebox{.1em}{$#1$}\left/\raisebox{-.1em}{$#2$}\right.}}

\newcommand{\RNum}[1]{\uppercase\expandafter{\romannumeral #1\relax}}


%%%%%%%%%%%%%%%%%%%%%%%%%%%%%%%%%%%%%%%%%
% Theorem 지정해주는 곳

\newtheoremstyle{break}
{\topsep}{\topsep}%
{\itshape}{}%
{\bfseries}{}%
{\newline}{}%
\theoremstyle{break}
\newtheorem{thm}{Theorem}[section] % reset theorem numbering for each chapter

\newtheoremstyle{newdef}% name
{}%         Space above, empty = `usual value'
{}%         Space below
{}%         Body font
{}%         Indent amount (empty = no indent, \parindent = para indent)
{\bfseries}%Thm head font
{}%        Punctuation after thm head
{\newline}% Space after thm head: \newline = linebreak
{}%         Thm head spec
\theoremstyle{newdef}
\newtheorem{defn}[thm]{Definition} % definition numbers are dependent on theorem numbers
\newtheorem{pro}[thm]{Proposition}
\newtheorem{cor}[thm]{Corollary}

\theoremstyle{remark}
\newtheorem*{exmp}{Example} % no numbering example
\newtheorem*{lem}{Lemma}
\newtheorem*{rem}{Remark} % no numbering remark



%%%%%%%%%%%%%%%%%%%%%%%%%%%%%%%%%%%%%%%%%

\begin{document}
\begin{titlepage}
\centering
{\scshape\LARGE Seoul National University \par}
\vspace{1cm}
{\scshape\Large Lecture Note\par}
\vspace{5.5cm}
{\huge\bfseries Introduction to Stochastic \\Differential Equations\par}
\vspace{1.5cm}
\large Lecture by Seo Insuk \\
Notes taken by Lee Youngjae


\vfill
\vspace{1cm}\par
{\large \today\par}
\end{titlepage}

\setlength{\parindent}{0cm}

\tableofcontents

\setcounter{chapter}{-1}

\chapter{Introduction}
E-mail: \textit{insuk.seo@snu.ac.kr, 27-212}\\
Office Hour: Tuesday 15:00 - 16:00

Grading
\begin{itemize}
\item Mid-terms 1 (15\%, 10/10 or 17)
\item Mid-terms 2 (15\%, 11/7)
\item Fianl-term (40\%)
\item Assignment (20\%, 8-10 times)
\item Attendance (10\%, absent: -2\%, late: -1\%)
\end{itemize}


Let $X$ be a standard normal random variable in $\R$.
i.e., $\mathbb{P}[X \in [a,b]] = \int_a^b \frac{1}{\sqrt{2\pi}} e^{-x^2/2}dx$.
(Central Limit Theorem) If $x_1, x_2, \cdots, x_n \in X, E(x_i) = m, Var(x_i) = \sigma^2$, then
$$
\frac{\frac{x_1-m}{\sigma}+\frac{x_2-m}{\sigma}+\cdots+\frac{x_n-m}{\sigma}}{\sqrt n} \rightarrow X
$$

In this class, we study dynamic version of this theorem.
If $(W_t)_{t\geq 0}$ be a fluctuation, then $(W_t)_{t\geq 0}$ be a random variable in $C[0,T]$

\begin{exmp}
$\frac{dX_t}{dt} = rX_t; dX_1 = rX_tdt$. Then, $X_t = X_0 e^{rt}$ (unrisky assets, bank)\\
$dX_t = rX_tdt + \sigma X_tdW_t$, $\sigma:$ volatility (risky assets, stock)
\end{exmp}

We will study:
\begin{enumerate}
\item Probability Space
\item Random Variable
\item Expectation
\end{enumerate}

Textbooks:
\begin{enumerate}
\item Stochastic Calculus for Finance \RNum{2} (Shreve), covering chapter 1-3 or 4
\item Introduction to Stochastic Integration (Hui-Hsiung Kuo)
\end{enumerate}



\chapter{General Probability Theory}
\section{Infinite Probability Spaces}
There are three elements consisting probability space:
\begin{itemize}
\item $S$: Sample space
\item $\mathcal{E}$: Family of events $\mathcal{E} \subset 2^S$ ($\sigma$-algebra in measure theory)
\item $\mathbb{P}$: probability $\Rightarrow \mathbb{P}(E)$ is defined for all $E \in \mathcal{E}$ ($\mu$ with $\mu(S)=1$)
\end{itemize}

\begin{exmp}
\begin{minipage}[t]{\linegoal}
\begin{enumerate}
\item Toss a coin twice (H for Head, T for Tail)\\
Then, $S = \{HH, HT, TT, TT\}$
\item Uniform random variable in $[0,1]^3$\\
Then, $S = [0,1]^3$.
If $E = [0,\frac{1}{2}]^3$, then $\mathbb{P}(E) = Vol(E) = \frac{1}{8}$\\
\end{enumerate}
\end{minipage}

How to define $\mathcal{E}$?\\
In example 2, let $\mathcal{E} = $ family of all subsets of $[0,1]^3$ naively.
But Banach-Tarski Paradox says there are disjoint sets $E,F$ with $\mathbb{P}(E\cup F) \neq \mathbb{P}(E) + \mathbb{P}(F)$ in this $\mathcal{E}$.
Therefore we cannot naively set $\mathcal{E}$ (Use measure theory)\\

In example 1, suppose that we cannot see the second flip.
If $\{HH\} \not\in \mathcal{E}$ and $\{HT, HH\}\in\mathcal{E}$, then $\mathcal{E} = \{\phi, \{HH,HT\}, \{TH,TT\}, \{HH,HT,TH,TT\}\}$
\end{exmp}


\begin{defn}[Measure]
Let $\Omega$ be a non-empty set and $\mathcal{F}$ be family of subsets of $\Omega$ with
\begin{enumerate}
\item $\phi \in \mathcal{F}$
\item $A \in \mathcal{F} \Rightarrow A^C \in \mathcal{F}$
\item $A_1, A_2, \cdots \in \mathcal{F} \Rightarrow \bigcup_{i=1}^\infty A_i \in \mathcal{F}$.
\end{enumerate}
We say $\mathcal{F}$ as \textbf{$\sigma$-algebra} or \textbf{$\sigma$-field}, $A \in \mathcal{F}$ as \textbf{measurable}, and $\Omega$ as \textbf{measurable space}.
\end{defn}


\textit{Exercises.}
\leavevmode
\begin{enumerate}[label = \arabic*)]
\item $\Omega \in \mathcal{F}$
\item $A_1, A_2, \cdots \in \mathcal{F}$, then $A_1\cap A_2 \cdots \in \mathcal{F}$
\item $A_1, A_2, \cdots \in A_n \in \mathcal{F}$, then $A_1 \cup \cdots \cup A_n, A_1 \cap \cdots \cap A_n \in \mathcal{F}$.
\item $A, B \in \mathcal{F}$, then $A-B \in \mathcal{F}$
\end{enumerate}


\begin{defn}[Topological Space]
(See Rudin: \textit{Real and Complex Analysis, Chapter 1.})
Let $\Theta$ be non-empty set and $\tau$ be family of subsets of $\Theta$ with
\begin{enumerate}
\item $\phi, \Theta \in \tau$
\item $V_1, \cdots V_n \in \tau \Rightarrow V_1 \cap \cdots \cap V_n \in \tau$
\item $V_\alpha \in \tau \enspace \forall \alpha \in I \Rightarrow \bigcup_{\alpha\in I}V_\alpha \in \tau$.
\end{enumerate}
We say $V\in\tau$ be an \textbf{open set}, and $(\Theta,\tau)$ be a \textbf{topological space}.
\end{defn}


\begin{defn}[Measurable Function]
$f : (\Omega, \mathcal{F}) \rightarrow (\Theta, \tau)$ is \textbf{measurable} if
$f^{-1}(V) \in \mathcal{F} \enspace \forall V \in \tau$
\end{defn}


\begin{defn}[Positive Measure]
Let $\Omega$ be non-empty set and $\mathcal{F}$ be $\sigma$-algebra.
Then $\mu: \mathcal{F} \rightarrow [0,\infty]$ is called \textbf{measurable} if
\begin{enumerate}
\item $A_1, A_2, \cdots$: disjoint members of $\mathcal{F} \Rightarrow \mu(A_1\cup A_2\cup \cdots) = \sum_{i=1}^\infty \mu(A_i)$
\item $\mu(A) < \infty$ for some $A \in \mathcal{F}$,
\end{enumerate}
and $(\Omega, \mathcal{F}, \mu)$ is called a \textbf{measure space}.
\end{defn}


\begin{defn}[probability space, random variable]
\leavevmode
\vspace{-6mm}
\begin{enumerate}
\item $(\Omega, \mathcal{F}, \mathbb{P})$ is called as \textbf{probability space} if $\mathbb{P}(\Omega) = 1$.
\item $X$ is called as \textbf{random variable} if it is a function from $(\Omega, \mathcal{F}, \mathbb{P})$ to $\mathbb{R}$
\end{enumerate}
\end{defn}

\underline{Next Class}
\begin{itemize}
\item Borel sets on $\mathbb{R}$ or $\mathbb{R}^d$
\item Lebesgue Measure
\item Lebesgue Integral (Define Expectation of random variable)
\end{itemize}

\vspace{5mm}

Last class, we define a sample space $\Omega$, a $\sigma$-algebra $\mathcal{F}$, and a (positive) measure $\mu : \mathcal{F} \rightarrow [0,\infty]$.
\vspace{5mm}

\textit{Exercises.}
\begin{itemize}
\item $A_1 \subset A_2 \subset \cdots \Rightarrow \mu(\bigcup_{i=1}^\infty A_i) = \lim_{n\rightarrow\infty}\mu(A_n)$
\item $A_1 \supset A_2 \supset \cdots, \mu(A_1) < \infty \Rightarrow \mu(\bigcap_{i=1}^\infty A_i) = \lim_{n\rightarrow\infty}\mu(A_n)$
\end{itemize}


\begin{thm}[Rudin 1.10]
Let $\mathcal{F}_0$ be a collection of subset of $\Omega$.
Then, $\exists! \mathcal{F}^*$ minimal $\sigma$-algebra containing $\mathcal F_0$.

\begin{proof}
Let $\{\mathcal{F}_\alpha, \alpha \in I\}$ be a family of $\sigma$-algebra containing $\mathcal{F}_0$.
Then, $\mathcal{F}^* = \bigcap_{\alpha\in I} F_\alpha$ satisfies the three condition:
1) contain $\mathcal{F}_0$
2) $\sigma$-algebra
3) minimal (trivial, $\mathcal{F}^* \subset \mathcal{F}_\alpha)$
\end{proof}
\end{thm}


\begin{defn}[Borel measurable]
$\mathcal{B}$ is a \textbf{Borel $\sigma$-algebra} on topological space $(\Theta,\tau)$ if
$\mathcal{B}$ is minimal $\sigma$-algebra containing $\tau$, and $B$ is a \textbf{Borel measurable} if $B \in \mathcal{B}$.
\end{defn}

\begin{rem}[Completion of measure space, Rudin 1.15]
\leavevmode\\
Consider an extension $(\Omega, \mathcal{F}, \mu) \rightarrow (\Omega, \overline{\mathcal{F}}, \mu)$ where
\begin{enumerate}
\item $\overline{\mathcal{F}} = \{ A \cup N : A \in \mathcal{F}, N \subset A_0 \in \mathcal{F}, \mu(A_0) = 0 \}$
\item $\mu(A \cup N) = \mu(A)$
\end{enumerate}
Then, (Check!)
\begin{enumerate}
\item (well-definedness) $A_1 \cup N_1 = A_2 \cup N_2 \Rightarrow \mu(A_1) = \mu(A_2)$
\item $\mu: \overline{\mathcal{F}}$ is $\sigma$-algebra.
\item $\mu : \overline{\mathcal{F}} \rightarrow [0,\infty]$ is a measure
\end{enumerate}
\end{rem}

\begin{exmp}
\leavevmode
\begin{enumerate}[label = \arabic*)]
\item $\mathbb{R}$
$$
\xymatrix@R+0.25em@C+2em{
\ar @{} [dr] |{}
\mathcal{F}_0 = \tau \ar[r]^{1.10} & \mathcal{B} \ar[r]^{\text{completion}} & \overline{\mathcal{B}}\\
\mathcal{L} \ar[r]^{\text{Rudin CH 2}} & \mathcal{L} \ar[r]^{\text{completion}} \ar[r] & \mathcal{L}
}
$$



\item $C[0,T] = \Omega = \{f ; f : [0,T] \rightarrow \mathbb{R}, \text{continuous}\}$.\\
Define $\mathcal{F}_0 = \{\bigcup_{t_1,t_2,\cdots,t_k}(A_1,A_2,\cdots,A_k)
: 0 \leq t_1 < t_2 < \cdots < t_k \leq T; A_1, \cdots A_k \in \overline{\mathcal{B}}
\}$.
We call $\{f \in C[0,T]: f(t_1) \in A_1, f(t_2) \in A_2, \cdots, f(t_k) \in A_k\}$ as \textbf{cylindrical set}.
Consider
% 깔끔하게좀 바꾸자
$$
\begin{aligned}
\mathcal{F}_0 &\overset{1.10}{\longrightarrow} &\mathcal{B} &\overset{\text{completion}}{\longrightarrow} &\overline{\mathcal{B}}\\
\mathbb{P}_{\text{BM}} &\overset{\text{KET}}{\longrightarrow} &\mathbb{P}_{\text{BM}} &\overset{\text{completion}}{\longrightarrow} &\mathbb{P}^*_{\text{BM}}
\end{aligned}
$$
(KET refers Kolmogorov's Extension Thm)
\end{enumerate}
\end{exmp}

\section{Random Variables and Distributions}
\begin{defn}
$f : \Omega \rightarrow \mathbb{R}$ is measurable if $f^{-1}(V) \in \mathcal{F}$ for any open set $V \subset \mathbb{R}$.
\end{defn}

\begin{rem}
$\mathcal{B}(\mathbb{R})$ = Borel $\sigma$-algebra in $\mathbb{R}$.
\end{rem}

\begin{rem}
If $f$: measurable, then $f^{-1}(B) \in \mathcal{F}$ for any $B \in \mathcal{B}(\mathbb{R})$.
\begin{proof}
Let $G = \{ A \subset \mathbb{R} : f^{-1}(A) \in \mathcal{F} \}$.
Then, $\tau \subset G$, $G: \sigma$-algebra (check!), hence $\mathcal{B}(\mathbb{R}) \subset G$.
\end{proof}
\end{rem}


\begin{defn}
\leavevmode
\vspace{-6mm}
\begin{itemize}
\item $(\Omega, \mathcal{F}, \mathbb{P})$ is a \textbf{probability space} if $(\mathbb{P}(\Omega) = 1$.
\item $X$ is \textbf{random variable} if $X : \Omega \rightarrow \mathbb{R}$ is measruable.
\end{itemize}
\end{defn}

\begin{exmp}
\leavevmode
\begin{enumerate}
\item Toss a coin Twice.\\
$\Omega = \{HH,HT,TH,TT\}$,
$\mathcal{F} = 2^\Omega = \{$all subsets of $\Omega\}$,
$\mathbb{P}(A) = \frac{1}{4}|A|, \enspace A \in \mathcal{F}$.\\
Then, $X = $ the number of H's is random variable with $X(HH) = 2, X(HT)=X(TH)=1, X(TT)=1$.

\item Uniform random variable in $[0,1]$\\
$\Omega=  [0,1]$,
$\mathcal{F} = \{B \in \mathcal{B}(\mathbb{R}) : B \subset [0,1]\}$,
$\mathbb{P}(B) = \mathcal{L}(B)$ ($\mathbb{P}([0,1]) = \mathcal{L}([0,1]) = 1$).\\
Then, $X : [0,1] \rightarrow \mathbb{R}$ with $X(x) = x$ be a (uniform) random variable in $[0,1]$.
\end{enumerate}
\end{exmp}


\begin{rem}
$\mathcal{L}$: Lebesgue measure on $\mathbb{R}$. i.e.,$\mathcal{L}(a,b) = b-a$.
Then, $\mathcal{L}(\{a\}) = 0$\\
$(\because
\{a\} = \bigcap_{i=1}^\infty (a-\frac{1}{n}, a+\frac{1}{n})
\Rightarrow \mathcal{L}(\{a\}) = \lim_{n\rightarrow\infty} \mathcal{L}((a-\frac{1}{n}, a+\frac{1}{n})) = 0
)$\\
Similarly, $\mathcal{L}([a,b]) = \mathcal{L}([a,b)) = \mathcal{L}((a,b]) = b-a$,
$\mathcal{L}(\mathbb{Q}) = \sum_{q \in \mathbb{Q}}\mathcal{L}(\{q\}) = 0$.
\end{rem}

Return to uniform random variable,
$$\mathbb{P}[X \in (a,b)] = \mathbb{P}[\{x : X(x) \in (a,b)\}] = \mathbb{P}[(a,b)] = b-a.$$

\begin{defn}[Distribution measure on $X$]
$X$ is a random variable in $(\Omega, \mathcal{F}, \mathbb{P})$.
$\mu_X$ is a \textbf{distribution measure} on $X$ if $\mu_X$ is a probability measure on $(\mathbb{R}, \mathcal{B}(\mathbb{R}))$ such that 
$$
\mu_X(B) = \mathbb{P}[X \in B]
= \mathbb{P}[\{\omega : X(\omega) \in B\}]
= \mathbb{P}[X^{-1}(B)] \ \forall B \in \mathcal{B}(\mathbb{R})
$$
\end{defn}


\begin{defn}[Probability density function]
$f$ is a \textbf{probability density function} of $X$ if $\mu_X((a,b)) = \int_a^b f(x)dx$
\end{defn}

\begin{rem}
There is a measure with no pdf: Dirac measure
\end{rem}

\begin{rem}
Lebesgue-Radon-Nikodym decomposition implies that any measure can be decomposed as density part and non-density part.
\end{rem}

\begin{exmp}[Standard Normal random variable]
\leavevmode\\
Let $\phi(x) = \frac{1}{\sqrt{2\pi}} e^{-x^2/2}$.
Define $F : (0,1) \rightarrow \mathbb{R}$ by $F(x) = N^{-1}(x)$ for $N(X) = \int_{-\infty}^x \phi(y)dy$.\\
Let $\Omega = (0,1), \mathcal{F} = \{B \in \mathcal{B}(\mathbb{R}) : B \subset (0,1)\}, \mathbb{P}(A) = \mathcal{L}(A) : A \in \mathcal{B}(\mathbb{R})$.\\
Then, $Y : \Omega \ni x \mapsto F(x) \in \mathbb{R}$ is a random variable with
$$
\begin{aligned}
\mathbb{P}[Y \in (a,b)]
&= \mathbb{P}[\{x : Y(x) \in (a,b)\}]\\
&= \mathbb{P}[\{x \in (N(a),N(b))\}]\\
&= N(b)-N(a) = \int_a^b \phi(x)dx,
\end{aligned}
$$
and a density function is $\phi$.
\end{exmp}


\vspace{5mm}
Previous Question: In the probability space $(\Omega, \mathcal{F}, \mathbb{P})$ and random variable $X : \Omega \rightarrow \mathbb{R}$, the random element or random realization $\omega \in \Omega$ is a element of events in sample space.
For example, $\omega = HHTTH$ is a random element in tossing a coin five times, and $X(\omega) = 3$. ($X(\omega)$ = \# of Heads)\\
In the previous example(Standard Normal random variable), define $(\Omega, \mathcal{F}, \mathbb{P}) = ((0,1), \mathcal{B}(0,1), \mathbb{P})$, $\mathbb{P}((a,b)) = b-a$, $F(x) = \int_{-\infty}^x \frac{1}{\sqrt{2\pi}}e^{-y^2/2}dy$, $X : (0,1) \ni \omega \mapsto F^{-1}(\omega) \in \mathbb{R}$.
Then, $X$ is called a standard normal random variable.







\section{Expectations}
In the following, let $\Omega = (\Omega, \mathcal{F}, \mathbb{P})$.
Let $X : \Omega \rightarrow \mathbb{R}$.
Then the expection $E(X)$ is a mean of $X(\omega)$ with respect to the randomness of $\omega$ (given by $\mathbb{P}$)


\begin{defn}[Lebesgue Integration]
$(\Omega, \mathcal{F}, \mu)$ is a measure space, and $f : \Omega \rightarrow \mathbb{R}$ is a measurable function.
\begin{enumerate}[label = (\arabic*)]
\item $f : \Omega \rightarrow [0,\infty)$\\
Let $0 = y_0 < y_1 < y_2 < \cdots \rightarrow \mathbb{R}$ be a partition of $[0,\infty)$,\\
$\Pi = \{y_0, y_1, y_2, \cdots\}$ : $\|\Pi\| = \sup_{i\geq 1}|y_i-y_{i-1}|$, and\\
$\text{LS}_\pi = \sum_{i=0}^\infty y_i \mu[f^{-1}([y_i,y_{i+1}))]$.\\
In Rudin's book, $\lim_{\|\Pi\|\rightarrow 0} \text{LS}_\pi$ converges to an element belonging to $[0,\infty]$.\\
Now, $\int fd\mu := \lim_{\|\Pi\|\rightarrow 0}\text{LS}_{\Pi}$ is called a \textbf{Lebesgue Integral}.

\item $f : \Omega \rightarrow \mathbb{R}$\\
Let $f^+ = \max\{f,0\} \geq 0$, and $f^- = -\min\{f,0\} \geq 0$.
Then, $f = f^+ - f^-$, and $|f| = f^+ + f^-$.
If $\int f^+d\mu < \infty$ and $\int f^- d\mu < \infty$, then we say $f$ is Lebesgue Integrable and $f \in L^1(\mu)$.
The Lebesgue Integral of $f$ $\int fd\mu$ is defined as $\int f^+ d\mu - \int f^- d\mu$
\end{enumerate}
\end{defn}

\begin{rem}
\leavevmode
\begin{enumerate}
\item $\int f^+ d\mu < \infty$ and $\int f^- d\mu = \infty$, then $\int fd\mu = -\infty$.
The others are defined similarly.
\item $f \in L^1(\mu) \Leftrightarrow \int |f|d\mu < \infty$.
\end{enumerate}
\end{rem}


\begin{exmp}[Riemann vs Lebesgue Integral (p19-22)]
\leavevmode
\begin{itemize}
\item $(\mathbb{R}, \mathcal{B}(\mathbb{R}), \mathcal{L})$ Lebesgue measure where $\mathcal{L}((a,b)) = b-a$.
\item $f : \mathbb{R} \rightarrow \mathbb{R} \in L^1(\mathcal{L})$
\item (Def) $A \subset \mathbb{R}$, $\int_A fd\mu := \int f \mathbb{1}_A d\mu$, where $\mathbb{1}_A(x) = 1$ if $x \in A$, and $0$ otherwise.
\end{itemize}
If $f$ is Riemann integrable, then $\int_{[a,b]} fd\mathcal{L} = \int_a^b fdx$.

Riemann integral is a limit of approximation by a partition of $x$-axis.
On the other hand, Lebesgue integral is a limit of approximation by a partition of $y$-axis with preimage.
Partition of $x$-axis is sensitive to fluctuation and restricted to Euclidean space, while partition of $y$-axis is not.
For example, $f(x) = \mathbb{1}_\mathbb{Q}(x)$ is Lebesgue integrable, but it is not Riemann integrable since it is sensitive to fluctuation.
\end{exmp}

\begin{defn}[Almost everywhere, 1.1.5 in Textbook]
$P(x)$ is a property at $x \in \mathbb{R}$.
We say $P$ holds \textbf{almost everywhere} (or a.e.) in $\mathbb{R}$ if and only if
$\mathcal{L}(\{x:P(x)$ does not hold $\} = 0$.
\end{defn}

\begin{exmp}
$f(x) = [x]$ is continuous almost everywhere.
\end{exmp}


\begin{thm}
$f$ is Riemann integrable if and only if $f$ is continuous a.e.
\end{thm}

\textit{Exercises.} $f = g$ a.e. $\Rightarrow \int fd\mathcal{L} = \int g d\mathcal{L}$.

\begin{defn}[Almost surely]
Let $(\Omega, \mathcal{F}, \mathbb{P})$ be a probability space.
The event $A (\in \mathcal{F})$ occusrs \textbf{almost surely} (a.s.) if $\mathbb{P}(A) = 1$.
\end{defn}

\begin{exmp}
Let $X$ be a uniform random variable in $(0,1)$.
Let $A = \{ X(\omega) \neq \frac{1}{2}\}$; $\mathbb{P}(A) = 1$.
\end{exmp}



\begin{defn}[Expectation, 1.3.3. in Textbook]
\textbf{Expectation} of $X : \Omega \rightarrow \mathbb{R}$ is defined by
$$E(X) := \int_\Omega X d\mathbb{P} \quad \text{if} \quad \int_{\Omega}|X|d\mathbb{P} < \infty$$ 
\end{defn}

\begin{thm}[1.3.4 in Textbook]
\leavevmode
\vspace{-6mm}
\begin{enumerate}
\item $X$ takes finite number of values $\{x_1,x_2, \cdots, x_n\} \Rightarrow E(X) = \sum_{i=1}^n x_i\mathbb{P}(X = x_i)$
\item $X, Y$: random variables, $E(|X|), E(|Y|) < \infty$,
\begin{enumerate}[label = (\roman*)]
\item $X \leq Y$ a.s. (i.e. $\mathbb{P}[\{X(\omega) \leq Y(\omega)\}]=1$), then $E(X) \leq E(Y)$
\item $X = Y$ a.s. $\Rightarrow E(X) = E(Y)$
\end{enumerate}
\item $X, Y$: random variables, $E(|X|), E(|Y|) < \infty \Rightarrow E(\alpha X + \beta Y) = \alpha E(X) + \beta E(Y)$.
\item Jensen's Inequality: $\phi : \mathbb{R} \rightarrow \mathbb{R}$ is a convex function $\Rightarrow \phi(E(X)) \leq E(\phi(X))$ (c.f. $\phi(t) = t^2$)
\end{enumerate}

\begin{proof}[Proof of 4.]
Define $S_\phi = \{(a,b) \in \mathbb{R}^2 : a + bt \leq \phi(t) \enspace \forall t\}$.
Then $\forall t \in \mathbb{R}, \phi(t) = \sup_{(a,b) \in S_\phi} \{a+bt\}$.
In fact, it is a equivalent condition. Now,
$$
\begin{aligned}
\phi(E(X)) &= \sup_{a,b\in S_\phi} \{a + bE(X)\}\\
&= \sup_{a,b \in S_\phi} E(a+bX)\\
&\leq E[\sup_{a,b\in S_\phi}(a+bx)] = E(\phi(X)) \quad (\text{Check!})
\end{aligned}
$$
\end{proof}
\end{thm}


\begin{exmp}[Dirac Measure in $\mathbb{R}$]
$(\mathbb{R}, \mathcal{B}(\mathbb{R}), \delta_y) \enspace (y \in \mathbb{R})$ is a probability space with $\delta_y(A) = 1$ if $y \in A$, and $0$ otherwise.
Then, $\int_\mathbb{R} fd\delta_y = f(y)$ (Check!)

\vspace{6mm}
Consider modeling: $X$: random variable such that probability of $x_i = p_i$ with $\sum_{i=1}^n p_i = 1$.
Then, $(\mathbb{R}, \mathcal{B}(\mathbb{R}), \mu)$ with $\mu = \sum_{i=1}^n p_i\delta_{x_i}$ is a probability space, and $P(X = x_i) = p_i$ for $X : \mathbb{R} \ni \omega \mapsto \omega \in \mathbb{R}$: Example of thm 1.3.4.
\end{exmp}



















\begin{comment}
나중에 따로 질문드릴 때 참고할 메모입니다. 강의와는 상관 없습니다.

9/5
이해 안되는거
cynlindrical set?
Kolmogorov's Extension Theorem?
distribution measure? (이해됨!)
LRN Thm imlies?

궁금한거
measure가 존재하면 distribution measure는 반드시 존재하는가?
distribution measure와 pdf는 어떤 관계가 있는가?

9/10
\end{comment}













\end{document}
